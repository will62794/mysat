\documentclass{beamer}
\usetheme{Boadilla}

\usepackage{tikz}

\title{SAT Solving with Conflict Driven Clause Learning}
% \subtitle{Overview and Implementation}
\author{William Schultz}
\institute{CS 7240 Final Project}
\date{\today}

\begin{document}

\newcommand{\green}[1]{\textcolor{green}{#1}}
\newcommand{\red}[1]{\textcolor{red}{#1}}

\begin{frame}
    \titlepage
\end{frame}

\begin{frame}{Overview and Project Goals}
\begin{itemize}
    \item Satisfiability is the canonical NP-complete problem.
    
    \item Much work has been devoted to building efficient SAT solvers over last decades.\\
    
    \item \textbf{Project Goal:} Implement a basic SAT solver based on \textit{conflict driven clause learning} (CDCL), which is the dominant core technique used in modern SAT solvers.
    \begin{itemize}
        \item Gain a deeper understanding of the DPLL and CDCL based algorithms for SAT solving
        \item Use as a platform for potentially exploring new SAT solving techniques and understanding limitations of existing ones
    \end{itemize}
\end{itemize}
\end{frame}

\begin{frame}{Review: The SAT Problem}
    The SAT problem:
    \vspace{12pt}

    \textit{Given a boolean formula in conjunctive normal form, determine whether there exists an assignment to the variables of the formula that makes the overall formula true.}

    \pause
    \vspace{12pt}
    e.g.
    \begin{align*}
        (x_1 \vee x_2) \wedge (\neg x_3 \vee \neg x_1)
    \end{align*}
    \pause
    \begin{center}
        SAT, with $\{x_1=1,x_2=0,x_3=0\}$. 
    \end{center}
\end{frame}

\begin{frame}{DPLL: SAT as Search}
    \begin{itemize}
        \item A basic approach to solving SAT is to view it as a search problem over possible assignments.
        \item This is the basis of the Davis–Putnam–Logemann–Loveland (DPLL) algorithm \cite{dpll1961}
        \item Basic idea of DPLL is to do a depth first, brute force search with backtracking along with some basic formula simplification as you go.
        \begin{itemize}
            \item Also employs the \textit{unit propagation rule}
        \end{itemize}
    \end{itemize}
\end{frame}

\begin{frame}{Unit Propagation}
    \begin{itemize}
        \item Core simplification rule employed in DPLL, and also in CDCL as we will see later.
        \item A \textit{unit clause} is a clause that contains exactly one literal.
        \item If a CNF formula contains a unit clause then we can apply unit propagation i.e. set that literal to the appropriate truth value to satisfy its clause e.g.
        \begin{align*}
            &\{\{b\}, \{\neg b, \neg c\}, \{c, \neg d\}\}\\
            \pause
            &\{\{\green{b}\}, \{\red{\neg b}, \neg c\}, \{c, \neg d\}\}\\
            \pause
            &\{\{\neg c\}, \{c, \neg d\}\} \\
            \pause
            &\{\{\green{\neg c}\}, \{\red{c}, \neg d\}\}\\
            \pause
            &\{\{\neg d\}\}
        \end{align*}
    \end{itemize}
\end{frame}

\begin{frame}{DPLL: Example}
    \begin{columns}
        % CNF formula.
        \begin{column}{0.45\textwidth}
            \begin{align*}
                &\{\neg a,b\}\\
                &\{\neg b,\neg c\}\\
                &\{\neg c,\neg d\}\\
            \end{align*}
        \end{column}

        % DPLL search tree.
        \begin{column}{0.45\textwidth}
            % \begin{center}
            \begin{tikzpicture}
                \node {a}
                child [] {node {b} 
                    child [] {node {c}
                        child [] {node {d} 
                            child [] {node {x} edge from parent [] node [left]{1}}
                            child [] {node {x} edge from parent [] node [right]{0}}
                        edge from parent [] node [left]{0}} 
                    edge from parent [] node [left]{1}}
                    child [] {node {c} 
                    edge from parent [] node [right]{0}}
                edge from parent [] node [left]{1}}
                child [] {node {b} 
                edge from parent [] node [right]{0}};
            \end{tikzpicture}        
            % \end{center}
        \end{column}
    \end{columns}
\end{frame}

\begin{frame}{Beyond DPLL: Learning from Conflicts}
    \begin{itemize}
        \item DPLL is a rather naive algorithm
        \item An extension to this basic framework is to \textit{learn from conflicts} 
        \item When you encounter a conflict in the search tree, \textit{learn} a clause that prevents you from making the similar mistakes again
        \item This fundamental approch is known as \textit{conflict-driven clause learning} (CDCL) and started being employed in SAT solvers around the late 90s and early 2000s.
        \item In addition, employ \textit{non-chronological backtracking}
    \end{itemize}
\end{frame}

\begin{frame}{CDCL}
\begin{itemize}
    \item When using CDCL, if a conflict is encountered, we not only backtrack to the previous level, as in DPLL
    \item We try to learn a \textit{conflict clause} along with a \textit{backjump} level, which determines how far back in the search tree to unwind to.
\end{itemize}
\end{frame}

\begin{frame}{CDCL: Example}
    \begin{align*}
        &\{a,b\}\\
        &\{b,c\}\\
        &\{\neg a, \neg x, y\} \\
        &\{\neg a, x, z\} \\
        &\{ \neg a, \neg y, z\} \\
        &\{ \neg a, x, \neg z\} \\
        &\{ \neg a, \neg y, \neg z\}
    \end{align*}
\end{frame}

\begin{frame}{SAT Solver Implementation}
    \begin{itemize}
        \item Worked on implementing my own CDCL SAT solver as a framework for exploring future potential SAT enhancements
        \item Around 1500 lines of C++, tested on a variety of easy to medium SAT benchmark problems
        \item \url{https://github.com/will62794/mysat}
    \end{itemize}
\end{frame}

\begin{frame}{Evaluation}
    \begin{itemize}
        \item Some peformance results of my SAT solver against a performant, modern solver.
    \end{itemize}
\end{frame}

\begin{frame}{Future Extensions}
    \begin{itemize}
        \item Resolution proofs
        \item Variable ordering heuristics
        \item Learning heuristics
        \item Learning end to end SAT solver (neuroSAT)
    \end{itemize}
\end{frame}

\bibliographystyle{alpha}
\bibliography{../references}

\end{document}

